%!TEX root = ../template.tex
%%%%%%%%%%%%%%%%%%%%%%%%%%%%%%%%%%%%%%%%%%%%%%%%%%%%%%%%%%%%%%%%%%%%
%% chapter5.tex
%% NOVA thesis document file
%%
%% Chapter with lots of dummy text
%%%%%%%%%%%%%%%%%%%%%%%%%%%%%%%%%%%%%%%%%%%%%%%%%%%%%%%%%%%%%%%%%%%%

\chapter{Conclusion and Future Work}
\label{cha:conclusion}

\section{Conclusion}
\paragraph{}This thesis presented the development, implementation, and validation of an autonomous navigation system for a tractor-trailer 
system to be used for pesticide spraying in agricultural environments. The work began by identifying the pressing need for 
increased efficiency and safety in agriculture, especially in the context of global food demand and the challenges 
posed by traditional farming methods. The proposed solution was designed to address the unique difficulties of 
tractor-trailer navigation, including non-holonomic constraints, modularity, and the need for collision 
safety in complex and constrained spaces.

\paragraph{}The literature review established the context for mobile robotics in smart agriculture, 
motion planning approaches, and control methods, highlighting the importance of the design of an efficient 
navigation system. The architecture chapter detailed the design, hardware selection, and system overview. The 
implementation chapter described the use of the \gls{ROS2} framework and \gls{NAV2} stack, which enabled modular 
development, and seamless integration of hardware and software components. The planning 
algorithm, combining Voronoi Hybrid A* and Dubins paths, provided feasible path generation, while the 
controller ensured accurate path tracking in the majority of scenarios.

\paragraph{}Testing in both simulated and real-world environments demonstrated the system's reliability, 
safety, and practical viability. The results showed that the robot could successfully navigate direct, obstructed, and 
recovery paths, as well as avoid collisions.

\paragraph{}Overall, the thesis achieved its objectives and contributed a solid foundation for autonomous agricultural 
robotics. The successful implementation and validation of the proposed system demonstrate its potential for real-world 
deployment, offering significant development in the study of solutions for the tractor-tailer system. 
The work also provides valuable insights and methodologies that can be extended to other agricultural and 
industrial applications.


\section{Future Work}
\label{sec:future_work}
\paragraph{}While the current system proved successful, there are several promising directions 
for future research and development. Integrating GPS-based localization would greatly enhance 
the robot's performance in large-scale agricultural fields, where map-based localization 
may be limited. This improvement would allow for more accurate global positioning and 
facilitate operations over extended areas, making the system even more versatile and scalable.

\paragraph{}Another important avenue is the redesign of the controller. The controller used 
in this work, while effective in many scenarios, demonstrated difficulty when navigating in consecutive 
changes in direction and curved maneuvers. By exploring more advanced control strategies 
such as \gls{MPC} or Machine learning approaches, the system could achive more precise 
control over its movements, leading to smoother and more reliable navigation.

\paragraph{}In summary, this thesis lays a foundation for autonomous 
agricultural robotics. The successful implementation and validation of the proposed 
system demonstrate its real-world potential, and the suggested future improvements offer 
exciting opportunities for further research and innovation in the field.

