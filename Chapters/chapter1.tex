%!TEX root = ../template.tex
%%%%%%%%%%%%%%%%%%%%%%%%%%%%%%%%%%%%%%%%%%%%%%%%%%%%%%%%%%%%%%%%%%%
%% chapter1.tex
%% NOVA thesis document file
%%
%% Chapter with introduction
%%%%%%%%%%%%%%%%%%%%%%%%%%%%%%%%%%%%%%%%%%%%%%%%%%%%%%%%%%%%%%%%%%%

\typeout{NT FILE chapter1.tex}%

\chapter{Introduction}
\label{cha:introduction}
\paragraph{}In this chapter, there will be a background explanation of the proposed problem, 
the motivation for a solution and the document structure.

\section{Background}
\label{sec:background} 
\paragraph{}Agriculture is one of the most essential industries, providing a source of food for the growing population 
around the globe. However, the industry faces several challenges, such as the need to increase food production, 
for example, according to the \gls{FAO} of the United Nations, around eight hundred million 
people are undernourished, and two thirds of them live in Asia. For example, in India, over 70\% of the population 
is dependent on agriculture for their livelihood ~\cite{agriIndia}, and, due to lack of the development of 
the countries’ agricultural processes, most farmers use very traditional methods of farming, which are not efficient 
requiring a lot of time and manual labour. To overcome these issues and to meet the growing demand for food,
the agriculture industry is turning to technology, such as robotics, to improve efficiency and productivity.
\paragraph{}The use of technologies like AI, Internet of Things, and robotics facilitates the automation of 
several tasks, such as planting, weeding, harvesting, forecasting, and monitoring which tackles the issues of 
labour shortages and allowing for easier expansion, solving the need for more food production. One particularly important application of robotics in agriculture is in pesticide spraying. 
Traditionally, pesticide application has been a labour-intensive and slow task. Autonomous pesticide 
spraying robots address these issues by applying pesticides more accurately, reducing chemical waste, and 
minimizing human exposure to harmful substances. 


\section{Motivation}
\label{sec:motivation}
\paragraph{}The motivation behind the development of a tractor-trailer pesticide spraying robot is the need to improve agricultural 
processes to meet the demand for food production and safety. By integrating these robots in Smart Farming, 
farmers can optimize pesticide usage, reducing the amount of pesticide wasted and overexposure to 
chemicals. The solution proposed in this work is to develop a trailer-tractor system, where the tractor will be responsible for 
towing a trailer with a pesticide spraying system. The advantages of this system are its modularity, with each part acting 
independently, allowing for easier reuse and maintenance, and also the fact that the development of the robot is not constrained by
the functionality of the system to be integrated, meaning that the tractor can be developed freely without having to account 
for the space occupied by the fluid tanks and balancing fluid tanks, leaving more room to develop both systems independently. 

The main technical issue with these systems is their non-holonomic 
characteristics, meaning that the tractor has its movement restricted due to the trailer's dynamics. A 
good example of a problem these systems can have is when a differential drive robot tries to rotate in place, the trailer will 
simply be dragged along, drifting sideways, where it might end up in a position that isn't the desired one or even capsizing. 
This dynamic makes planning and control very challenging since both the planning and control systems will have to account for 
the constraint on the turning radius of the tractor-trailer system when calculating a path and sending motion commands. 

When analysing the advantages and disadvantages of the proposed system, it is clear that, with a good path planning
and motion control solution, the advantages far outweigh the disadvantages making the proposed solution a viable  
approach for increasing safety and efficiency in smart agriculture.

\section{Document Structure}
\label{sec:documentstruct}
\paragraph{}This document is divided into 4 chapters. The first chapter is the introduction, 
where the problem is presented along with the motivation for the work and the document structure. 
The second is the state of the art review, where a contextualization of mobile robotics in agriculture is made 
along with the most popular methods for path planning and robot control, and some previous work 
on the trailer-tractor system. The third chapter is the planning and schedule of the work, 
where the work proposal, schedule and results publishing plan are presented. The last chapter is the 
conclusion, where a summary of the document is presented.