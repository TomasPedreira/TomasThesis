%!TEX root = ../template.tex
%%%%%%%%%%%%%%%%%%%%%%%%%%%%%%%%%%%%%%%%%%%%%%%%%%%%%%%%%%%%%%%%%%%%
%% abstract-en.tex
%% NOVA thesis document file
%%
%% Abstract in English([^%]*)
%%%%%%%%%%%%%%%%%%%%%%%%%%%%%%%%%%%%%%%%%%%%%%%%%%%%%%%%%%%%%%%%%%%%

\typeout{NT FILE abstract-en.tex}%

\paragraph{}

Over the years, the world of technology has been expanding in every 
possible field, and agriculture is no exception. The use of technology in Smart 
Agriculture has been increasing, with Internet of Things networks, Artificial 
Intelligence systems for management and forecasting, and robotics to automate 
processes, performing repetitive jobs humans would rather not do, such as applying 
pesticides, harvesting products, and evaluating the state of the fields, as well as 
tasks humans cannot easily accomplish, like surveilling large fields. These developments
 not only facilitate tasks but also address ongoing challenges in the agricultural 
 sector, including the shortage of young people willing to undertake demanding field 
 work and the growing global population, which will heighten the demand for food 
 production. This thesis presents the development of a tractor-trailer robotic system 
 capable of autonomously navigating through a field. It includes a review of 
 robotic applications in Smart Agriculture and an exploration of popular robot path planning 
 methods like the Voronoi Graph, A* grid search along with others. This work also reviews several control methods 
 including Dynamic Window Approach, Pure Pursuit, and Model 
 Predictive Control. This task is inherently challenging, and the addition of a 
 trailer to the system increases the complexity, as the trailer's dynamics must be 
 accounted for. The selected methods implemented and tested were the Voronoi Graph 
 combined with the Hybrid A* algorithm for path planning and the Pure Pursuit controller for 
 trajectory tracking. The system was architected with hardware components, custom 
 planner and controller plugins in ROS2 and NAV2 stack and collision detection mechanisms. 
 Tests conducted in simulation and real-world 
 environments evaluated path tracking in direct, obstructed, and recovery scenarios, 
 as well as obstacle detection, demonstrating the system's effectiveness in navigation 
 and safety while highlighting areas for improvement. The tractor-trailer system was 
 chosen, despite its complexity and limited documentation, due to its modularity, 
 enabling independent operation and reusability.

\keywords{
  Smart Agriculture \and
  tractor-trailer \and
  Path planning \and
  Trajectory tracking \and
  Robot control \and
  ROS2 \and
  NAV2 \and
  Voronoi Graph \and
  Hybrid A* \and
  Pure Pursuit 
}
