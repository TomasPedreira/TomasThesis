%!TEX root = ../template.tex
%%%%%%%%%%%%%%%%%%%%%%%%%%%%%%%%%%%%%%%%%%%%%%%%%%%%%%%%%%%%%%%%%%%%
%% abstract-pt.tex
%% NOVA thesis document file
%%
%% Abstract in Portuguese
%%%%%%%%%%%%%%%%%%%%%%%%%%%%%%%%%%%%%%%%%%%%%%%%%%%%%%%%%%%%%%%%%%%%

\typeout{NT FILE abstract-pt.tex}%

\paragraph{}Ao longo dos anos, a tecnologia tem-se expandido em todos 
os campos, e a agricultura não é exceção. A utilização na Agricultura 
Inteligente tem aumentado, com redes de Internet das Coisas, sistemas de
Inteligência Artificial para gestão e previsão, e robótica para 
automatizar processos, realizando tarefas repetitivas que humanos 
preferem evitar, como aplicar pesticidas, colher produtos e avaliar 
campos, além de tarefas difíceis como vigiar grandes áreas. Estes 
desenvolvimentos facilitam tarefas e abordam desafios no setor 
agrícola, como a escassez de jovens para trabalhos de campo exigentes e o 
crescimento da população global, intensificando a demanda por alimentos. 
Esta tarefa é desafiante, e a adição de um atrelado aumenta a 
complexidade, pois as dinâmicas do atrelado devem ser consideradas. O 
sistema tractor-trailer foi escolhido, apesar da complexidade e 
documentação limitada, pela sua modularidade, permitindo operação 
independente e reutilização. Esta tese apresenta o desenvolvimento de 
um sistema robótico tractor-trailer capaz de navegar autonomamente num 
campo. Inclui revisão de aplicações robóticas na Agricultura Inteligente e 
exploração de métodos de planeamento de trajetórias para robôs, como o 
Voronoi Graph, A* juntamente com outros. Revê também métodos de seguimento de trajetórias, 
incluindo Dynamic Window Approach, Pure Pursuit e Model Predictive Control. 
Os métodos selecionados, implementados e testados, foram o Voronoi Graph 
combinado com Hybrid A* para planeamento de trajetórias e Pure Pursuit 
para seguimento de trajetória. O sistema foi arquitetado com plugins personalizados 
de planeador e controlador na stack ROS2 e NAV2, e mecanismos de 
deteção de colisões. Testes em simulação e no mundo real avaliaram o 
seguimento de trajetórias em cenários diretos, obstruídos e de 
recuperação, além da deteção de obstáculos, demonstrando eficácia na 
navegação e segurança, destacando áreas para melhoria.

\keywords{ Agricultura Inteligente \and 
tractor-trailer \and
Planeamento de trajetórias \and
Seguimento de trajetória \and
ROS2 \and
NAV2 \and
Voronoi Graph \and 
Hybrid A* \and 
Pure Pursuit 
}
% to add an extra black line
