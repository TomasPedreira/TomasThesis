%!TEX root = ../template.tex
%%%%%%%%%%%%%%%%%%%%%%%%%%%%%%%%%%%%%%%%%%%%%%%%%%%%%%%%%%%%%%%%%%%%
%% abstract-pt.tex
%% NOVA thesis document file
%%
%% Abstract in Portuguese
%%%%%%%%%%%%%%%%%%%%%%%%%%%%%%%%%%%%%%%%%%%%%%%%%%%%%%%%%%%%%%%%%%%%

\typeout{NT FILE abstract-pt.tex}%

\paragraph{}Ao longo dos anos, o mundo da tecnologia tem-se expandido em todos os 
campos, e a agricultura não é exceção. O uso de tecnologia na agricultura tem vindo a aumentar, com redes de Internet of Things, sistemas de 
Inteligência Artificial para gestão e previsão, e robótica para automatizar processos, 
realizando tarefas repetitivas que os humanos preferem não fazer, como aplicar 
pesticidas, colher produtos e avaliar o estado dos campos, bem como tarefas que os 
humanos não conseguem realizar facilmente, como vigiar grandes campos. Estes 
desenvolvimentos facilitam tarefas e abordam desafios no 
setor agrícola, incluindo a escassez de jovens dispostos a realizar trabalhos exigentes 
no campo e o crescente aumento da população global, que irá intensificar a procura pela 
produção de alimentos. Esta tese apresenta o desenvolvimento de um sistema 
tractor-trailer capaz de navegar autonomamente. Inclui uma revisão de 
aplicações robóticas na Agricultura Inteligente e uma exploração de métodos populares de 
planeamento de trajetórias como o Voronoi Graph e o A* grid search e suas variantes juntamente com outros. Este 
trabalho também revê vários métodos de controlo, incluindo Dynamic Window Approach, 
Pure Pursuit e Model Predictive Control. Esta tarefa é inerentemente desafiante, e a 
adição de um trailer ao sistema aumenta a complexidade, pois as dinâmicas do trailer 
devem ser consideradas. Os métodos selecionados implementados e testados foram o Voronoi 
Graph combinado com o algoritmo Hybrid A* para planeamento de trajetórias e o controlador Pure 
Pursuit para seguimento de trajetória. O sistema foi arquitetado com componentes de 
hardware, plugins personalizados de planner e controller no stack ROS2 e NAV2 e 
mecanismos de deteção de colisões. Testes realizados em ambientes de simulação e 
no mundo real avaliaram o seguimento de trajetória em cenários diretos, obstruídos e de 
recuperação, bem como a deteção de obstáculos, demonstrando a eficácia do 
sistema em navegação e a sua segurança, ao mesmo tempo que destacam áreas para melhoria. 
O sistema tractor-trailer foi escolhido, apesar da sua complexidade e documentação 
limitada, devido à sua modularidade, permitindo operação independente e reutilização.


\keywords{ Agricultura Inteligente \and 
tractor-trailer \and
Planeamento de trajetórias \and
Seguimento de trajetória \and
Controlo \and 
ROS2 \and
NAV2 \and
Voronoi Graphs \and 
Hybrid A* \and 
Pure Pursuit 
}
% to add an extra black line
